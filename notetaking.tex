\documentclass{article}
\usepackage{rdfref-user}
\usepackage{marginnote}

% #1 citekey for the publication
\makeatletter
\newcommand\source[1]{%
	% define blank node prefix, this should enable using notes from 
	% multiple files
	\def\rdf@blk{#1}
	\BlankNode
	\AddProperty{rdf:type}{ann:publication}
	\AddProperty{ann:citekey}{#1}
	\edef\notepagepub{\CurrentObject}%
}

% pagu number or some other identifier
\newcommand\notepage[1]{%
	\BlankNode%
  \AddProperty{rdf:type}{ann:pageNo}
	\AddPropertyEx{ann:hasSource}{\notepagepub}
	\AddPropertyEx{ann:hasIdentifier}{#1}
	\edef\notepagelabel{\CurrentObject}% Save current 
}

\begin{document}
\title{Book reading notetaking system with rdfref}

\source{bib:pepa14}
\notepage{1}
Co je třeba

- definovat zpracovávanou publikaci
- odkazovat na definovatelný pasáže (stránky, nadpisy, čísla sekcí)
- v makru uchovávat label současný pasáže
- v poznmkách a citacích se odkazovat na label pasáže
- poznámky samostatný autoimatický labely nebo labely v nepovinných parametrech
- odkazy na otázky, slovník atd se vztahují k poznámkám

\reversemarginpar
Pokus\marginnote{[1]}
\end{document}

