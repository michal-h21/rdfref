\documentclass{ltxdoc}
\usepackage{rdfref}
\usepackage[english]{babel}
\usepackage{hyperref}
\usepackage{fontspec}
\usepackage{csquotes}
\usepackage{showexpl}
\setmainfont{TeX Gyre Schola}
\setmonofont[Scale=MatchLowercase]{Inconsolatazi4}
\usepackage{microtype}
\title{The \texttt{rdfref} package}
\author{Michal Hoftich\footnote{\href{mailto:michal.h21@gmail.com}{michal.h21@gmail.com}}}
\date{Version \version\\\gitdate}
\begin{document}
\maketitle
\tableofcontents

\section{Introduction}

\href{https://en.wikipedia.org/wiki/Resource_Description_Framework}{RDF (Resource Description Framework)}
is a standard model for representing and
exchanging data on the web. It provides a flexible way to describe resources
such as people, places, and things, as well as their relationships and
attributes.

RDF data is structured as triples, which consist of a subject, a predicate, and
an object. The subject represents the resource being described, the predicate
represents the relationship or attribute being described, and the object
represents the value of that relationship or attribute. For example, a triple
might describe a person (the subject) having a name (the predicate) of \enquote{John
Smith} (the object).

Rdfref is a package that brings some of the RDF concepts to \LaTeX. It allows
setting properties to different objects, which can then be retrieved using a
query language based on various criteria. Objects can be, for example,
chapters, equations, or images. Properties can be page numbers, counter values,
captions, and so on.

The initial impulse for the creation of this package was the need to create a
dependency graph between individual elements of a document, such as which
chapters reference individual theorems. However, over time the package has
evolved to be useful for other purposes as well.

\section{Tutorial}


\subsection{Basic concepts}

Objects and properties typically have the form "namespace:name".

This naming convention helps to avoid naming conflicts between different
objects and properties, and allows for more organized and modular code. The
namespace portion of the name identifies the context or domain to which the
object or property belongs, while the name portion provides a unique identifier
within that context.

Let's say we want to assign the property \enquote{person:name} with the value \enquote{Michal}
to an object with the identifier \enquote{authors:michal}. 


In this case, the identifier \enquote{authors:michal} serves as a unique
identifier for an object in a specific context. The identifier
belongs to an object that represents a person named \enquote{michal} in namespace \enquote{authors}.

The \enquote{person} namespace identifies the context of the propety as
being related to personal information, and the \enquote{name} property specifies a
specific attribute of that object. 

In this case, the value associated with the \enquote{person:name} property is the
string \enquote{Michal}. In other cases, the value associated with a property may also
be an identifier that references another object. This is a common pattern in
data modeling and can be used to establish relationships between objects.


By using this naming convention consistently
throughout your code, you can ensure that your objects and properties are
well-organized and easy to manage.





\subsection{Low-level functions}


We can model the example from the previous section using the following code:

\begin{LTXexample}[pos=b]
  \AssignProperty{authors:michal}{person:name}{Michal}
  My name is: \GetProperty{authors:michal}{person:name}

\end{LTXexample}


\subsection{Query Language}

\subsection{Add-ons for \LaTeX\ cross-referencing}


\end{document}

