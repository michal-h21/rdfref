\documentclass{ltxdoc}
\usepackage{rdfref}
\usepackage[english]{babel}
\usepackage{hyperref}
\usepackage{fontspec}
\usepackage{csquotes}
\usepackage{showexpl}
\setmainfont{TeX Gyre Schola}
\setmonofont[Scale=MatchLowercase]{Inconsolatazi4}
\usepackage{microtype}
\title{The \texttt{rdfref} package}
\author{Michal Hoftich\footnote{\url{michal.h21@gmail.com}}}
\date{Version \version\\\gitdate}
\begin{document}
\maketitle
\tableofcontents

\section{Introduction}

\href{https://en.wikipedia.org/wiki/Resource_Description_Framework}{RDF (Resource Description Framework)}
is a standard model for representing and
exchanging data on the web. It provides a flexible way to describe resources
such as people, places, and things, as well as their relationships and
attributes.

RDF data is structured as triples, which consist of a subject, a predicate, and
an object. The subject represents the resource being described, the predicate
represents the relationship or attribute being described, and the object
represents the value of that relationship or attribute. For example, a triple
might describe a person (the subject) having a name (the predicate) of \enquote{John
Smith} (the object).

Rdfref is a package that brings some of the RDF concepts to \LaTeX. It allows
setting properties to different objects, which can then be retrieved using a
query language based on various criteria. Objects can be, for example,
chapters, equations, or images. Properties can be page numbers, counter values,
captions, and so on.


\section{Tutorial}

\subsection{Low-level functions}

\begin{LTXexample}
  % set 
  \SetProperty{person:michal}{name}{Michal}
  My name is: \GetProperty{person:michal}{name}

\end{LTXexample}

\subsection{Query Language}

\subsection{Add-ons for \LaTeX\ cross-referencing}


\end{document}

