\documentclass{article}
\usepackage[utf8]{inputenc}
\usepackage[]{tgschola}
\usepackage[T1]{fontenc}
\usepackage{nameref}
\usepackage{hyperref}
\usepackage{tikz}
\usepackage[]{rdfref-user,rdfref-query,lipsum,nameref}
\usepackage{amsthm,thmtools,xstring}
\declaretheorem[name=Lemma,Refname={Lemma,Lemmas}]{lem}
\declaretheorem[name=Proposition,Refname={Proposition,Proposition}]{prop}
\declaretheorem[name=Theorem,Refname={Theorem,Theorems}]{thr}

\begin{document}
\inputontology{theorem-ontology.tex}
\let\ref\rdfref
\section{Hello world}\rdflabel{sec:first}
\lipsum[1-2]
  \begin{lem}\rdflabel{thm:lem1}
    \partOf{thm:thr1}
    A lemma. Reference to theorem \ref{thm:euclid}
  \end{lem}

  \begin{prop}
    \rdflabel{thm:prop1}
    \partOf{thm:thr1}
    A proposition.
    %\renewcommand{\theenumi}{\thecor.\arabic{enumi}}
    \begin{enumerate}
      \item An item 
      \item Another one 
    \end{enumerate}
  \end{prop}

  \begin{thr}\rdflabel{thm:thr1}
    A theorem.
  \end{thr}

\begin{thr}[Euclid]
\rdflabel{thm:euclid}%
For every prime $p$, there is a prime $p’>p$.
In particular, the list of primes,
\begin{equation}\rdflabel{eq:1}
2,3,5,7,\dots
\end{equation}
is infinite.
\end{thr}

As it was said in theorem \ref{thm:euclid}, we see that 

\forcsvlist{\listadd\typelist}{sec:sectioning,thm:theorem,eq:equation,thm:proposition,thm:proposition,thm:lemma}
\forcsvlist{\listadd\proplist}{doc:pageNo,doc:partOf,doc:hasParent}
\Bind{?x}{rdf:type}{?c}{%
 \xifinlist{\GetVal{?c}}{\typelist}{%
  \Bind{?x}{?k}{?v}{%
   \Bind{?x}{rdfs:label}{?objectl}{%
   \Bind{?c}{rdfs:label}{?typel}{% 
    \xifinlist{\GetVal{?k}}{\proplist}{%
     \IfProperty{\GetVal{?v}}{rdf:type}{\def\ObjectLabel{\GetProperty{\GetProperty{\GetVal{?v}}{rdf:type}}{rdfs:label}}}{\def\ObjectLabel{\GetVal{?v}}}
     \typeout{Object \GetVal{?objectl} : \GetVal{?typel} \detokenize{has key }\GetVal{?k} with value: \ObjectLabel}%
    }{} 
   }}%
  }%
 }%
 {\typeout{Object \GetVal{?x} of type \GetVal{?c} is not contained in typelist}}%
}
%\thepage
\newpage
And now print theorem info:

\typeout{Tak co sec?: \GetProperty{sec:sectioning}{rdf:type}}

% Make list of classes, which are subclasses of thm:theorem
% use query to add type #2 to list #1
% this helper list will be used to loop over all theorems
\newcommand\buildtypelist[2]{%
	\listadd#1{#2} % add searched 
	\Bind{?x}{rdfs:subClassOf}{#2}{%
		% only direct children of #2 are taken into account
		\listxadd#1{\GetVal{?x}}% must be global and expanded
	}%
}

% build list of theorems
\buildtypelist\theoremlist{thm:theorem}

\newcommand\MakeLabel[1]{%
	\Bind{#1}{rdf:type}{?type}{% get type of parent
		\textbf{\GetValProperty{?type}{rdfs:label}} % parent type label
	}% 
	% now clickable reference to parent
	\hyperref[\GetVal{#1}]{\GetValProperty{#1}{rdfs:label}}%
}



\renewcommand\do[1]{%
  % loop over all objects that have rdf:type of the argument
  \Bind{?obj}{rdf:type}{#1}{%
    % Print basic info about the object
    \par Object type: 
    \textbf{\GetProperty{#1}{rdfs:label}}, % #1 is theorem type label 
    % now clickable reference to label of current object
    value:\textbf{\hyperref[\GetVal{?obj}]{\GetValProperty{?obj}{rdfs:label}}},
    % and just label
    label: [\GetVal{?obj}].
    % Print info about parent object:
    \Bind{?obj}{doc:hasParent}{?par}{% parent label
      \par \hspace{2em}Parent object:%
      \MakeLabel{?par}%	
    }% print objects that reference the current theorem
    \def\refinfo{\par\hspace{2em} Referenced by:\par\hspace{3em}}
    \Bind{?ref}{doc:refersTo}{?obj}{%
      \refinfo\def\refinfo{\par\hspace{3em}}%
      \GetValProperty{?ref}{rdfs:label} in
      \Bind{?ref}{doc:hasParent}{?par}{%
        \MakeLabel{?par}%
      }, p. \hyperpage{\GetValProperty{?ref}{doc:pageNo}};%
    }%
  }
}

% this will execute \do macro for each element in the \theoremlist
\dolistloop\theoremlist

% Now export a theorem graph

% Make expandable label
\newcommand\MakePlainLabel[1]{%
  \IfProperty{#1}{rdf:type}{\GetProperty{\GetProperty{#1}{rdf:type}}{rdfs:label}}{}
	\GetProperty{#1}{rdfs:label}
}

\makeatletter
% Make list of used nodes, every node should be added only once
% #1 list csname #2 node prefix #3 subject
\newcommand\AddNode[3]{%
	\ifcsdef{node@lbl@#2@#3}{}{%
		\listxadd#1{#3}
		\csgdef{node@lbl@#2@#3}{\MakePlainLabel{#3}}
	}%
}

% replace colons with underscores
\def\coltoun#1:#2;{#1_#2}% graphwiz doesn't support colon in names 

\def\graphrelations{}
\def\mynodes{}

% connect nodes
% #1 source node #2 destination #3 graphviz style
\newcommand\AddRelation[3]{%
	\edef\tempa{#1}%
	\edef\tempb{#2}%
	% add node label. we can call \AddNode for node multiple times,
	% node is beeing added only once
	\AddNode\mynodes{myn}{#1}%
	\AddNode\mynodes{myn}{#2}%
	% store relations in a macro
	\xdef\graphrelations{%
		\graphrelations \expandafter\coltoun\tempa; -> \expandafter\coltoun\tempb; [#3];^^J
	}%
}

\renewcommand\do[1]{%
	\Bind{?obj}{rdf:type}{#1}{%
		% select parent node. for lemmas etc, which use doc:partOf property, 
		% use that, otherwise use doc:has:parent
		\IfProperty{\GetVal{?obj}}{doc:partOf}{%
			\edef\parnode{\GetValProperty{?obj}{doc:partOf}}
		}{
			\edef\parnode{\GetValProperty{?obj}{doc:hasParent}}
		}
		\AddRelation{\GetVal{?obj}}{\parnode}{}
		\Bind{?ref}{doc:refersTo}{?obj}{%
			% select references to a theorem and print parent subject
			\def\parnode{\GetValProperty{?ref}{doc:hasParent}}
			\AddRelation{\parnode}{\GetVal{?obj}}{style=dotted}
		}
	}
}

% again loop over al theorems
\dolistloop\theoremlist

% now prepare output file for the graphviz output
\newwrite\graphwrite

% macros for opening and writing to the graph file
\newcommand\graphopen[1]{%
	\immediate\openout\graphwrite=#1%
}

\newcommand\graphout[1]{\immediate\write\graphwrite{#1}}


\graphopen{\jobname.dot}
\graphout{digraph hello\@charlb} % save graph header

% save labels
\renewcommand\do[1]{%
	\graphout{\coltoun#1; [label="\MakePlainLabel{#1}"]}
}
\dolistloop\mynodes

% save graph relations
\graphout{\graphrelations}
\graphout{\@charrb} % save footer
\immediate\closeout\graphwrite

\begin{figure}

\InputIfFileExists{theorems-sample-graph.tex}{}{}
\caption{Dependency graph of theorems}
\end{figure}

\end{document}

