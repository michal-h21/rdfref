\documentclass{article}
\usepackage[utf8]{inputenc}
\usepackage[]{tgschola}
\usepackage[T1]{fontenc}
%\catcode`\_=12
\usepackage[]{rdfref-user,rdfref-query,lipsum,nameref,cleveref}
\usepackage{amsthm,thmtools,xstring}
\declaretheorem[name=Lemma,Refname={Lemma,Lemmas}]{lem}
\declaretheorem[name=Proposition,Refname={Proposition,Proposition}]{prop}
\declaretheorem[name=Theorem,Refname={Theorem,Theorems}]{thr}

\begin{document}
\thepage \thepage
\inputontology{theorem-ontology.tex}
\let\ref\rdfref
\thepage
\section{Hello world}\rdflabel{sec:first}
\lipsum[1-2]
  \begin{lem}\rdflabel{thm:lem1}
    \partOf{thm:thr1}
    A lemma. Reference to theorem \ref{thm:euclid}
  \end{lem}

  \begin{prop}
    \rdflabel{thm:prop1}
    \partOf{thm:thr1}
    A proposition.
    %\renewcommand{\theenumi}{\thecor.\arabic{enumi}}
    \begin{enumerate}
      \item An item 
      \item Another one 
    \end{enumerate}
  \end{prop}

  \begin{thr}\rdflabel{thm:thr1}
    A theorem.
  \end{thr}

\begin{thr}[Euclid]
\rdflabel{thm:euclid}%
For every prime $p$, there is a prime $p’>p$.
In particular, the list of primes,
\begin{equation}\rdflabel{eq:1}
2,3,5,7,\dots
\end{equation}
is infinite.
\end{thr}

As it was said in theorem \rdfref{thm:euclid}, we see that 

\forcsvlist{\listadd\typelist}{sec:sectioning,thm:theorem,eq:equation,thm:proposition,thm:proposition,thm:lemma}
\forcsvlist{\listadd\proplist}{doc:pageNo,doc:partOf,doc:hasParent}
\Bind{?x}{rdf:type}{?c}{%
 \xifinlist{\GetVal{?c}}{\typelist}{%
  \Bind{?x}{?k}{?v}{%
   \Bind{?x}{rdfs:label}{?objectl}{%
   \Bind{?c}{rdfs:label}{?typel}{% 
    \xifinlist{\GetVal{?k}}{\proplist}{%
     \IfSubStr{\GetVal{?v}}{:}{\edef\ObjectLabel{\GetProperty{\GetProperty{\GetVal{?v}}{rdf:type}}{rdfs:label}}}{\edef\ObjectLabel{\GetVal{?v}}}
     \typeout{\GetVal{?objectl} : \GetVal{?typel} \detokenize{má klíč }\GetVal{?k} s hodnotou \ObjectLabel}%
    }{} 
   }}%
  }%
 }%
 {\typeout{Objekt \GetVal{?x} neni v listu}}%
}
\thepage

\typeout{Tak co sec?: \GetProperty{sec:sectioning}{rdf:type}}

% Make list of objects, which are type of thm:theorem

% use query to add type #2 to list #1
\newcommand\buildtypelist[2]{%
	\listadd#1{#2}
	\Bind{?x}{rdf:type}{#2}{%
		\listxadd#1{\GetVal{?x}}% must be global and expanded
	}%
}
\buildtypelist\theoremlist{thm:theorem}
\Bind{?ref}{rdf:type}{doc:reference}{%
	\edef\ReferedObj{\GetProperty{\GetVal{?ref}}{doc:refersTo}}
	\edef\ReferedType{\GetProperty{\ReferedObj}{rdf:type}}
	\Bind{?ref}{doc:refersTo}{?obj}{
	  \typeout{refered type:\ReferedType, label\GetValProperty{?obj}{rdfs:label}}
  }
}


\renewcommand\do[1]{%
\Bind{?obj}{rdf:type}{#1}{%
\textbf{\GetValProperty{?obj}{rdfs:label}}
				\typeout{sss: \GetVal{?obj} - \GetValProperty{?obj}{rdfs:label}} 
}}
\dolistloop\theoremlist

\end{document}
